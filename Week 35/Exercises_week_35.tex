\documentclass[12pt,
               a4paper,
               article,
               oneside,
               norsk,oldfontcommands]{memoir}
\usepackage{student}
% Metadata
\date{\today}
\setmodule{FYS-STK4155: Applied Data Analysis and Machine Learning}
\setterm{Fall, 2023}

%-------------------------------%
% Other details
% TODO: Fill these
%-------------------------------%
\title{Weekly Exercises: Week 35}
\setmembername{Jonas Semprini Næss}  % Fill group member names

%-------------------------------%
% Add / Delete commands and packages
% TODO: Add / Delete here as you need
%-------------------------------%
\makeatletter
\newcommand*{\rom}[1]{\expandafter\@slowromancap\romannumeral #1@}
\makeatother
%\usepackage[utf8]{inputenc}
\usepackage{setspace}
\usepackage[T1]{fontenc}
\usepackage{titling}% the wheel somebody else kindly made for us earlier
\usepackage{fancyhdr}
\usepackage{fancybox}
\usepackage{epigraph} 
\usepackage{tikz}
\usepackage{pgfplots}
\pgfplotsset{compat=1.12}
\usepackage{lmodern}
\usepackage{enumitem}
\usepackage{caption}
\usepackage{subcaption}
\usepackage{fancyvrb}
\usepackage[scaled]{beramono}
\usepackage[final]{microtype}
\usepackage{amssymb}
\usepackage{mathtools}
\usepackage{amsthm}
\usepackage{thmtools}
\usepackage{babel}
\usepackage{csquotes}
\usepackage{listings}
\usetikzlibrary{calc,intersections,through,backgrounds}
\usepackage{tkz-euclide} 
\lstset{basicstyle = \ttfamily}
\usepackage{float}
\usepackage{textcomp}
\usepackage{siunitx}
\usepackage{xcolor}
\usepackage{graphicx}
\usepackage[colorlinks, allcolors = uiolink]{hyperref}
\usepackage[noabbrev]{cleveref}
\pretolerance = 2000
\tolerance    = 6000
\hbadness     = 6000
\newcounter{probnum}[section]
\newcounter{subprobnum}[probnum] 
\usepackage{dirtytalk}
\usepackage{listings}
\usepackage{xcolor}
\usepackage{caption}
\usepackage[section]{placeins}
\usepackage{varwidth}
\usepackage{optidef}
\definecolor{uiolink}{HTML}{0B5A9D}
\definecolor{dkgreen}{rgb}{0,0.6,0}
\definecolor{gray}{rgb}{0.5,0.5,0.5}
\definecolor{mauve}{rgb}{0.58,0,0.82}
\lstset{frame=tb,
  language=R,
  aboveskip=3mm,
  belowskip=3mm,
  showstringspaces=false,
  columns=fullflexible,
  basicstyle={\small\ttfamily},
  numbers=none,
  numberstyle=\tiny\color{gray},
  keywordstyle=\color{blue},
  commentstyle=\color{dkgreen},
  stringstyle=\color{mauve},
  breaklines=true,
  breakatwhitespace=true,
  tabsize=3
} 
\usepackage{commath}
\newtheorem{theorem}{Theorem}[section]
\newtheorem{corollary}{Corollary}[theorem]
\newcommand{\Q}{ \qquad \hfill \blacksquare}
\newcommand\myeq{\stackrel{\mathclap{\normalfont{uif}}}{\sim}}
\let\oldref\ref
\renewcommand{\ref}[1]{(\oldref{#1})}
\newtheorem{lemma}[theorem]{Lemma}
\setlength \epigraphwidth {\linewidth}
\setlength \epigraphrule {0pt}
\AtBeginDocument{\renewcommand {\epigraphflush}{center}}
\renewcommand {\sourceflush} {center}
\parindent 0ex
\renewcommand{\thesection}{\roman{section}} 
\renewcommand{\thesubsection}{\thesection.\roman{subsection}}
\newcommand{\KL}{\mathrm{KL}}
\newcommand{\R}{\mathbb{R}}
\newcommand{\E}{\mathbb{E}}
\newcommand{\T}{\top}
\newcommand{\bl}{\left\{}
\newcommand{\br}{\right\}}
\newcommand{\spaze}{\vspace{4mm}\\}
\newcommand{\N}{\mathbb{N}}
\newcommand{\Rel}{\mathbb{R}}
\newcommand{\expdist}[2]{%
        \normalfont{\textsc{Exp}}(#1, #2)%
    }
\newcommand{\expparam}{\bm \lambda}
\newcommand{\Expparam}{\bm \Lambda}
\newcommand{\natparam}{\bm \eta}
\newcommand{\Natparam}{\bm H}
\newcommand{\sufstat}{\bm u}

% Main document
\begin{document}
\header{}
\section*{ \centering Exercise 1: Analytical Exercises}
a.) \emph{Show that}
\begin{align*}
\frac{\partial (\mathbf{b}^T \mathbf{a})}{\partial \mathbf{a}} = \mathbf{b}.
\end{align*}
\textbf{Solution:} \spaze 
By convention we assume that both $\mathbf{a},\mathbf{b} \in \R^{n \times 1}$. Hence can we re-write the given expression as 
\begin{align*}
\frac{\partial \left( \sum_{i} b_i a_i \right)}{\partial a_k}
\end{align*}
\footnote{I have decided to adapt standard typing and not boldface for specifying when we are working with scalar elements of respective vectors or matricies}
$ \forall i,k = 0, 1, 2, \ldots, n-1$, which yields 
\begin{align*}
\frac{\partial \left( \sum_{i} b_i a_i \right)}{\partial \mathbf{a_k}}  &= b_k \\[5pt] & \Downarrow \\[5pt]
\frac{\partial (\mathbf{b}^T \mathbf{a})}{\partial \mathbf{a}} &= \begin{bmatrix} b_0 \\ b_2 \\ \vdots \\ b_{n-1} \end{bmatrix} = \mathbf{b} 
\end{align*}
$\Q$. \spaze 
b.) \emph{Show that} 
\begin{align*}
\frac{\partial (\mathbf{a}^T \mathbf{A} \mathbf{a})}{\partial \mathbf{a}} = \mathbf{a}^T( \mathbf{A} + \mathbf{A}).
\end{align*}
Let $\mathbf{W} = \mathbf{A} \mathbf{a} $ such that the original quadratic form now becomes $  \mathbf{a}^T  \mathbf{W}$. By this we can initiate the product rule for derivatives. Thus 
\begin{align*}
\frac{\partial \mathbf{a}^T}{\partial \mathbf{a}} \mathbf{W} + \mathbf{a}^T \frac{\partial \mathbf{W}}{\partial \mathbf{a}}
\end{align*}
which by applying (a.) gives that 
\begin{align*}
\frac{\partial \mathbf{a}^T}{\partial \mathbf{a}} \mathbf{W} + \mathbf{a}^T \frac{\partial \mathbf{W}}{\partial \mathbf{a}} = \mathbf{W}^T + \mathbf{a}^T\mathbf{A}.
\end{align*}
Finally exploit the mathematical fact of $ \boxed{(AB)^T = B^TA^T}$ yielding 
\begin{align*}
\mathbf{W}^T + \mathbf{a}^T \mathbf{A} &= \mathbf{a}^T \mathbf{A}^T + \mathbf{a}^T \mathbf{A} \\[5pt] 
&= \mathbf{a}^T (\mathbf{A} + \mathbf{A}^T)
\end{align*}
which is what we wanted to show. $\Q$ \spaze 
c.) \emph{Show that} 
\begin{align*}
\frac{\partial \left(\mathbf{x}-\mathbf{A}\mathbf{s}\right)^T\left(\mathbf{x}-\mathbf{A}\mathbf{s}\right)}{\partial \mathbf{s}} = -2\left(\mathbf{x}-\mathbf{A}\mathbf{s}\right)^T\mathbf{A}.
\end{align*}
We follow the same reasoning as in (b.) which gives us the following 
\begin{align*}
\frac{\partial \left(\mathbf{x}-\mathbf{A}\mathbf{s}\right)^T}{\partial \mathbf{s}} \left(\mathbf{x}-\mathbf{A}\mathbf{s}\right) + \left(\mathbf{x}-\mathbf{A}\mathbf{s}\right)^T \frac{\partial \left(\mathbf{x}-\mathbf{A}\mathbf{s}\right)}{\partial \mathbf{s}}
\end{align*}.
Split up the parentheses such that 
\begin{align*}
\frac{\partial \mathbf{x}^T}{\partial \mathbf{s}} \left(\mathbf{x}-\mathbf{A}\mathbf{s}\right) - \left( \frac{\partial \left(\mathbf{A}\mathbf{s}\right)^T}{\partial \mathbf{s}} \right) \left(\mathbf{x}-\mathbf{A}\mathbf{s}\right) + 
\left(\mathbf{x}-\mathbf{A}\mathbf{s}\right)^T \frac{\partial \mathbf{x}}{\partial \mathbf{s}} - \left(\mathbf{x}-\mathbf{A}\mathbf{s}\right)^T \left(\frac{\partial \left(\mathbf{A}\mathbf{s}\right)}{\partial \mathbf{s}} \right).
\end{align*}
Then differentiate with respect to $\mathbf{s}$, yielding 
\begin{align*}
- \mathbf{A}( \mathbf{x} - \mathbf{A}\mathbf{s})^T - ( \mathbf{x} - \mathbf{A}\mathbf{s})^T \mathbf{A}  = -2( \mathbf{x} - \mathbf{A}\mathbf{s})^T  \mathbf{A}.
\end{align*}
Which is what we wanted to show. $\Q$ \spaze 
d.) \spaze 
To analyse the second derivate of the above mentioned result, we first differentiate $-2( \mathbf{x} - \mathbf{A}\mathbf{s})^T  \mathbf{A}$ with respect to $\mathbf{s}$ again. This gives: 
\begin{align*}
-2 \frac{\partial (\mathbf{x} - \mathbf{A}\mathbf{s})^T  \mathbf{A}}{\partial \mathbf{s}} &= 2 \frac{(\mathbf{A} \mathbf{s})^T}{\partial \mathbf{s}} \mathbf{A} \\[5pt]
&= 2 \mathbf{A}^T \mathbf{A}
\end{align*}
Which describes the Hessian Matrix. If $\mathbf{A}$ happens to be symmetric then obviously we would have that $-2 \frac{\partial (\mathbf{x} - \mathbf{A}\mathbf{s})^T  \mathbf{A}}{\partial \mathbf{s}} = 2 \mathbf{A}^2$, and if orthogonal then $-2 \frac{\partial (\mathbf{x} - \mathbf{A}\mathbf{s})^T  \mathbf{A}}{\partial \mathbf{s}} = 2 $. $\Q$
\end{document}