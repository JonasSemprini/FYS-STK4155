\documentclass[12pt,
               a4paper,
               article,
               oneside,
               norsk,oldfontcommands]{memoir}
\usepackage{student}
% Metadata
\date{\today}
\setmodule{FYS-STK4155: Applied Data Analysis and Machine Learning}
\setterm{Fall, 2023}

%-------------------------------%
% Other details
% TODO: Fill these
%-------------------------------%
\title{Weekly Exercises: Week 38}
\setmembername{Jonas Semprini Næss}  % Fill group member names

%-------------------------------%
% Add / Delete commands and packages
% TODO: Add / Delete here as you need
%-------------------------------%
\makeatletter
\newcommand*{\rom}[1]{\expandafter\@slowromancap\romannumeral #1@}
\makeatother
%\usepackage[utf8]{inputenc}
\usepackage{setspace}
\usepackage[T1]{fontenc}
\usepackage{titling}% the wheel somebody else kindly made for us earlier
\usepackage{fancyhdr}
\usepackage{fancybox}
\usepackage{epigraph} 
\usepackage{tikz}
\usepackage{bm}
\usepackage{pgfplots}
\pgfplotsset{compat=1.12}
\usepackage{lmodern}
\usepackage{enumitem}
\usepackage{caption}
\usepackage{subcaption}
\usepackage{fancyvrb}
\usepackage[scaled]{beramono}
\usepackage[final]{microtype}
\usepackage{amssymb}
\usepackage{mathtools}
\usepackage{amsthm}
\usepackage{thmtools}
\usepackage{babel}
\usepackage{csquotes}
\usepackage{listings}
\usetikzlibrary{calc,intersections,through,backgrounds}
\usepackage{tkz-euclide} 
\lstset{basicstyle = \ttfamily}
\usepackage{float}
\usepackage{textcomp}
\usepackage{siunitx}
\usepackage{xcolor}
\usepackage{graphicx}
\usepackage[colorlinks, allcolors = uiolink]{hyperref}
\usepackage[noabbrev]{cleveref}
\pretolerance = 2000
\tolerance    = 6000
\hbadness     = 6000
\newcounter{probnum}[section]
\newcounter{subprobnum}[probnum] 
\usepackage{dirtytalk}
\usepackage{listings}
\usepackage{xcolor}
\usepackage{caption}
\usepackage[section]{placeins}
\usepackage{varwidth}
\usepackage{optidef}
\definecolor{uiolink}{HTML}{0B5A9D}
\definecolor{dkgreen}{rgb}{0,0.6,0}
\definecolor{gray}{rgb}{0.5,0.5,0.5}
\definecolor{mauve}{rgb}{0.58,0,0.82}
\lstset{frame=tb,
  language=R,
  aboveskip=3mm,
  belowskip=3mm,
  showstringspaces=false,
  columns=fullflexible,
  basicstyle={\small\ttfamily},
  numbers=none,
  numberstyle=\tiny\color{gray},
  keywordstyle=\color{blue},
  commentstyle=\color{dkgreen},
  stringstyle=\color{mauve},
  breaklines=true,
  breakatwhitespace=true,
  tabsize=3
} 
\usepackage{commath}
\newtheorem{theorem}{Theorem}[section]
\newtheorem{corollary}{Corollary}[theorem]
\newcommand{\Q}{ \qquad \hfill \blacksquare}
\newcommand\myeq{\stackrel{\mathclap{\normalfont{uif}}}{\sim}}
\let\oldref\ref
\renewcommand{\ref}[1]{(\oldref{#1})}
\newtheorem{lemma}[theorem]{Lemma}
\setlength \epigraphwidth {\linewidth}
\setlength \epigraphrule {0pt}
\AtBeginDocument{\renewcommand {\epigraphflush}{center}}
\renewcommand {\sourceflush} {center}
\parindent 0ex
\renewcommand{\thesection}{\roman{section}} 
\renewcommand{\thesubsection}{\thesection.\roman{subsection}}
\newcommand{\KL}{\mathrm{KL}}
\newcommand{\R}{\mathbb{R}}
\newcommand{\E}{\mathbb{E}}
\newcommand{\T}{\top}
\newcommand{\bl}{\left\{}
\newcommand{\br}{\right\}}
\newcommand{\spaze}{\vspace{4mm}\\}
\newcommand{\N}{\mathbb{N}}
\newcommand{\Rel}{\mathbb{R}}
\newcommand{\expdist}[2]{%
        \normalfont{\textsc{Exp}}(#1, #2)%
    }
\newcommand{\expparam}{\bm \lambda}
\newcommand{\Expparam}{\bm \Lambda}
\newcommand{\natparam}{\bm \eta}
\newcommand{\Natparam}{\bm H}
\newcommand{\sufstat}{\bm u}

% Main document
\begin{document}
\header{}
\section*{\centering Analytical Exercises:}
\textbf{(\rom{1}.)} \emph{The parameters $\boldsymbol{\beta}$ are in turn found by optimizing the mean
squared error via the so-called cost function}
$$
C(\boldsymbol{X},\boldsymbol{\beta}) =\frac{1}{n}\sum_{i=0}^{n-1}(y_i-\tilde{y}_i)^2=\mathbb{E}\left[(\boldsymbol{y}-\boldsymbol{\tilde{y}})^2\right].
$$
\emph{Here the expected value $\mathbb{E}$ is the sample value.}
\emph{Show that you can rewrite  this in terms of a term which contains the variance of the model itself (the so-called variance term), a
term which measures the deviation from the true data and the mean value of the model (the bias term) and finally the variance of the noise.
That is, show that} 
$$
\mathbb{E}\left[(\boldsymbol{y}-\boldsymbol{\tilde{y}})^2\right]=\mathrm{Bias}[\tilde{y}]+\mathrm{var}[\tilde{y}]+\sigma^2,
$$
\textbf{Solution:} \spaze
We start of by rewriting $\mathbb{E}\left[(\boldsymbol{y}-\boldsymbol{\tilde{y}})^2\right] $ to $ \mathbb{E}\left[(\boldsymbol{f} + \boldsymbol{\epsilon} -\boldsymbol{\tilde{y}})^2\right]$. From this one gets that 
\begin{align*}
        \mathbb{E}[(\boldsymbol{y}-\boldsymbol{\tilde{y}})^2] &= \mathbb{E}[(\bm{f} + \bm{\varepsilon} - \bm{\Tilde{y}})^2] \\[5pt]
        &= \mathbb{E}[(\bm{f} - \bm{\tilde{y}}) + \bm{\varepsilon})^2] \\[5pt]
        &= \mathbb{E}[(\bm{f}-\bm{\Tilde{y}})^2 + 2(\bm{f}-\bm{\Tilde{y}})\bm{\varepsilon} + \bm{\varepsilon}^2] \\[5pt]
        &= \mathbb{E}[(\bm{f}-\bm{\Tilde{y}})^2] + 2\mathbb{E}[(\bm{f}-\bm{\Tilde{y}})\underbrace{\bm{\varepsilon}}_{=0}]+\underbrace{\mathbb{E}[\bm{\varepsilon}^2]}_{= \sigma^2} \\
        &= \mathbb{E}[(\bm{f}-\bm{\Tilde{y}})^2] + \sigma^2 .
\end{align*}
By adding and subtracting $\mathbb{E}[\bm{\Tilde{y}}]$ we obtain
\begin{align*}
         \mathbb{E}[(\bm{y}-\bm{\tilde{y}})^2] &= \mathbb{E}[(\bm{f}-\bm{\Tilde{y}}+ \mathbb{E}[\bm{\tilde{y}}] - \mathbb{E}[\bm{\tilde{y}}])^2]+\sigma^2 \\[5pt]
        &= \mathbb{E}[((\bm{f}- \mathbb{E}[\bm{\tilde{y}}]) + (\mathbb{E}[\bm{\tilde{y}}] -\bm{\Tilde{y}}))^2]+\sigma^2\\[5pt]
        & =\mathbb{E}[(\bm{f}- \mathbb{E}[\bm{\tilde{y}}])^2 + 2(\bm{f}- \mathbb{E}[\bm{\tilde{y}}])(\mathbb{E}[\bm{\tilde{y}}] -\bm{\Tilde{y}}) + (\mathbb{E}[\bm{\tilde{y}}]-\bm{\Tilde{y}})^2] + \sigma^2\\[5pt]
        &= \mathbb{E}[(\bm{f}- \mathbb{E}[\bm{\tilde{y}}])^2 + (\mathbb{E}[\bm{\tilde{y}}]-\bm{\Tilde{y}})^2] + 2\mathbb{E}[(\bm{f}- \mathbb{E}[\bm{\tilde{y}}])(\mathbb{E}[\bm{\tilde{y}}] -\bm{\Tilde{y}})] + \sigma^2
\end{align*}
where
\begin{equation}
    \begin{aligned}
        \mathbb{E}[(\bm{f}- \mathbb{E}[\bm{\tilde{y}}])( \mathbb{E}[\tilde{\bm{y}} -\bm{\Tilde{y}})] &= \mathbb{E}[\bm{f}\mathbb{E}[\bm{\tilde{y}}] - \bm{f\Tilde{y}} - \mathbb{E}[\bm{\tilde{y}}]^2 + \bm{\Tilde{y}}\mathbb{E}[\bm{\tilde{y}}]]\\[5pt]
        &= \bm{f}\mathbb{E}[\bm{\tilde{y}}] - \bm{f}\mathbb{E}[\bm{\tilde{y}}] - \mathbb{E}[\bm{\tilde{y}}]^2 + \mathbb{E}[\bm{\tilde{y}}]^2\\[5pt]
        &= 0
    \end{aligned}
\end{equation}
Thus, are we left with
\begin{align*}
         \mathbb{E}[(\bm{y}-\bm{\tilde{y}})^2] &= \mathbb{E}[(\bm{f}- \mathbb{E}[\bm{\tilde{y}}])^2 + (\mathbb{E}[\bm{\tilde{y}}]-\bm{\Tilde{y}})^2] + \sigma^2\\[5pt]
         &= \underbrace{\mathbb{E}[(\bm{f}- \mathbb{E}[\bm{\tilde{y}}])^2]}_{\text{Bias}[\tilde{y}]} + \underbrace{\mathbb{E}[(\bm{\Tilde{y}}-\mathbb{E}[\bm{\tilde{y}}])^2]}_{\text{Var}[\tilde{y}]} + \sigma^2\\[5pt]
         &= \text{Bias}[\tilde{y}] + \text{Var}[\tilde{y}] + \sigma^2.
\end{align*}
Which is what we wanted to show. $\Q$
\end{document}